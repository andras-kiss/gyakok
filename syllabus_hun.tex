\documentclass{article}
\usepackage{amsmath}
\usepackage{amssymb}
\usepackage{gensymb}
\usepackage{upgreek}
\usepackage{float}
\usepackage{graphicx}
\usepackage[T1]{fontenc}
\usepackage[utf8]{inputenc}
\usepackage[hungarian]{babel}

%\hyphenation{fortran er-go-no-mic}



\begin{document}

\title{Fontos információk a  gyógyszerész hallgatók Fizikai Kémia gyakorlatához}
%\author{Andr\'{a}s Kiss, assistant lecturer}
%Department of General and Physical Chemistry, Faculty of Sciences, University of P\'{e}cs, 7624 P\'{e}cs, Ifj\'{u}s\'{a}g \'{u}tja 6, Hungary
\date{2016.09.07}
\maketitle
%\address[akiss, gnagy]{J\'{a}nos Szent\'{a}gothai Research Centre, University of P\'{e}cs, 7624 P\'{e}cs, Ifj\'{u}s\'{a}g \'{u}tja 20, Hungary}
%\ead{akiss@gamma.ttk.pte.hu}

A tárgy 12 laborgyakorlatból és két zh-ból áll. Javasolt az alapos felkészülés minden egyes gyakoraltra. A felkészültséget ellenőrizni fogjuk egy 20 perces írásbeli ,,beugró'' teszttel, melyben a témához kapcsolódó néhány kérdést kell megválaszolnia. Ha ezekre nem tud megfelelően válaszolni, nem kezdheti el a gyakorlatot (sikertelen gyakorlat), melyet a 14. héten pótolhat. A gyakorlat alatt rögzítenie kell mérési eredményeit egy erre szolgáló jegyzőkönyvbe, nem-radírozható tollal. Csak a jegyzőkönyvébe írjon, ne külön lapokra. A jegyzőkönyvének elsősorban teljesnek és naprakésznek kell lennie, és csak aztán szépnek. Ha belejavít, egyetlen vonallal húzza át, hogy a hibás bejegyzés utólag is olvasható legyen. A gyakorlat végén aláírást kap az adataira a gyakorlatvezetőtől. Ne távozzon enélkül, különben nem fogja tudni bizonyítani, hogy a saját adatait használta.

Továbbá, a gyakorlat végeztével ki kell értékelnie a mérési adatait, kiszámolnia a végeredményeket (otthon), és egy előre megbeszélt határidő előtt leadnia. Csak a saját adatai használhatja, senki másét. Minden gyakorlat jegyzőkönyve pontozva lesz 0 - 5 skálán. A szemeszter végén összesen 60 pont szerezhető gyakorlati jegyzőkönyvekből.

Két zh-t fog írni: a 7. és a 13. héten. Ezekkel 12-12 pont szerezhető, ha jól megválaszolja a 6-6 gyakorlathoz tartozó kérdést, beleértve a zh írásának napján végzett gyakorlatot is. 

A félév elfogadásának feltétele a legalább 50\% megszerzése mindkét zh-ra, egyenként. Ezen kívül feltétel az összes gyakorlatra legalább 1 pont szerzése. Ha elvégzi a gyakorlatot, de nem értékeli ki, arra a gyakorlatra 0 pontot kap.

Egy gyakorlatról hiányozhat, amit utána pótolnia kell a 14. héten. A zh-kat is újraírhatja ezen a héten.

Összefoglalva:

\begin{enumerate}
\item 60 pont gyakorlat + jegyzőkönyvekre (12 $\times$ 5 pont).  Mindegyik legalább 1 pont.
\item 12 pont az első zh-ra. Legalább 50\% szükséges az elfogadáshoz.
\item 12 pont a második zh-ra. Legalább 50\% szükséges az elfogadáshoz. 
\end{enumerate}

Ha ezeknek a feltételeknek eleget tesz, a félév végi jegye az összpontszáma alapján lesz megállapítva:

\begin{center}
\begin{tabular}{|r|l|}
\hline
pontok & érdemjegy \\
\hline
0 -- 41 & 1 (elégtelen) \\
\hline
42 -- 53 & 2 (elégséges) \\
\hline
54 -- 63 & 3 (közepes) \\
\hline
64 -- 73 & 4 (jó) \\
\hline
74 -- 84 & 5 (jeles) \\
\hline
\end{tabular}
\end{center}

A zh-kat a 7. és a 13. hét péntekén írják (2016.10.21 és 2016.12.02) 16:00 órakor. Pótló alkalom egy gyakorlathoz és a két zh-hoz: a 14. hét pénteke (2016.12.09).

Jegyzőköny leadás: előre megbeszélt időpontban az Általános és Fizikai Kémia tanszék (Ifjúság útja 6.) folyosójának üvegajtós szekrényében. Javítás és pontozás után tudja elvinni jegyzőkönyvét a következő gyakorlatra való felkészüléshez.

Vegye komolyan a határidőt, mert a késedelmes leadásért pontlevonás jár.

\end{document}
