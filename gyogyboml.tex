\documentclass[a4paper, 12pt, twoside]{article}
\usepackage[hungarian]{babel}
\usepackage{ae,aecompl}
\usepackage[T1]{fontenc}
\usepackage[utf8]{inputenc}
\usepackage{textcomp}
\usepackage{pgfplots}
\usepackage{anysize}
\marginsize{3.2cm}{2.8cm}{3cm}{2cm}
\usepackage{setspace}
\setstretch{1.2}
\frenchspacing
\pgfplotsset{compat=newest}
%\pagestyle{empty}
\usepgfplotslibrary{ternary}
\usepackage{chemfig}
\usepackage{gensymb}
\usepackage{fancyhdr}

\hyphenation{hő-mér-sék-let-füg-gé-sét}

\renewcommand\footrule{\begin{minipage}{1\textwidth}
\hrule width \hsize height 2pt \kern 1mm \hrule width \hsize   
\end{minipage}\par}%

\renewcommand\headrule{
\begin{minipage}{1\textwidth}
\hrule width \hsize \kern 1mm \hrule width \hsize height 2pt 
\end{minipage}}%
\pagestyle{fancy}
\fancyhf{}
\fancyhead[LE,RO]{\leftmark}
\fancyfoot[LE,RO]{\thepage}
\fancyfoot[RE,LO]{\emph{Fizikai Kémia gyakorlatok gyógyszerész hallgatóknak}}

%\pagestyle{fancy}
%\fancyhf{}
%\usepackage{lastpage}
%\rfoot{Page \thepage \hspace{1pt} of \pageref{LastPage}}
%\lhead{\thesection}
%\cfoot{\itshape\textcolor{gray}{Fizikai Kémia gyakorlatok gyógyszerészeknek}}

\begin{document}

\setcounter{section}{2}
\section{Gyógyszerbomlás sebességének hőmérsékletfüggése}
\subsection{Bevezetés}

A gyakorlat során az \emph{Aspirin} (acetilszalicilsav) hidrolízisének kinetikailag elsőrendű reakciójának hőmérsékletfüggését vizsgáljuk.
A sebességi állandója a következőképpen adható meg:

\begin{equation}
\label{eq:divider}
        k
        =
        \frac
                {1}
                {t}
	\ln
	\frac{z}{z-x}
\end{equation}

ahol $t$ az idő, $z$ a reagens (jelen esetben a \emph{Aspirin}) kezdeti koncentrációja, $x$ pedig az elbomlott reagens koncentrációja.

A reakció sebessége vagy a sebességi állandó értéke függ a hőmérséklettől.
A hőmérsékletfüggést az \emph{Arrhenius egyenlet} írja le:

\begin{equation}
\label{eq:divider}
        \frac
                {d\ln k}
                {dT}
	=
	\frac
		{E}
		{\mathrm{R}T^2}
\end{equation}

melynek integrált alakja:

\begin{equation}
\label{eq:divider}
        k
        =
	A
	e^{-E/( \mathrm{R} T)}
\end{equation}

illetve

\begin{equation}
\label{eq:divider}
        \lg k
        =
        \lg A
	-\frac{E}{2.303 \mathrm{R}T}
\end{equation}

Az egyenletben $A$ a preexponenciális tényező, $E$ az aktiválási energia, és R a gázállandó (R$ = 8.314$ J/Kmol).
Az aktiválási energia meghatározható grafikus úton, ha az $\lg k - 1/T$ függvény meredekségét megmérjük és azt szorozzuk 2.303 $\times$ 8.314-el, amikor az $E$-t J/molban kapjuk meg.
Ha két hőmérsékleten megmérjük a reakciósebességi együtthatót ($k_1$-t és $k_2$-t $T_1$ és $T_2$ hőmérsékleten) az aktiválási energia a következő képlettel számítható ki:

\begin{equation}
	E
	=
	2.303
	\times
	8.314
	\lg
	\frac{k_1}{k_2}
	\frac{T_1 T_2}{T_1-T_2}
\end{equation}

\subsection{A gyakorlat kivitelezése}
Az \emph{Aspirin} hidrolízise kinetikailag elsőrendű folyamat és az alábbiak szerint játszódik le:

\begin{figure}
\centering
\schemedebug{false}
\schemestart
	\footnotesize \chemname{\chemfig{*6(-=-(-O-[::-60]([::-60]=O)-)=(-(-[::-60]OH)=[::60]O)-=)}}{Acetilszalicilsav}
	\footnotesize \+
	\footnotesize \chemfig{OH^{-}}\arrow(.mid east--.mid west){->[k][]}
	\footnotesize \chemname{\chemfig{*6(-=-(-OH)=(-([::-60]-OH)=[::60]O)-=)}}{Szalicilsav} + CH$_3$COO$^-$
\schemestop
\caption{Az acetilszalicilsav lúgos hidrolízise.}
\label{fig:salicilsav}
\end{figure}

A reakció szobahőmérsékleten igen lassú, ezért a méréseket magasabb hőmérsékleten végezzük.
A reakció sebességi együtthatójának meghatározásához ismerni kell a reaktáns vagy a termék koncentrációjának változását a reakcióidővel.
Jelen reakcióban a képződő szalicilsav Fe$^{3+}$ ionokkal alkotott stabil ibolyaszínű komplexét határozzuk meg spektrofotometriás módszerrel.
A lúgos közegben lejátszódó reakcióelegyből meghatározott reakcióidőnél ismert mennyiségű mintákat veszünk, a reakciót befagyasztjuk a hőmérséklet és a [OH$^-$] hirtelen csökkentésével.
Az előírt hígításokat követően a szalicilsav Fe(III)-komplexének koncentrációját spektrofotometriás úton meghatározzuk. Higításra lehet szükség, ha az abszorbancia 2 feletti, ekkor ugyanis a legtöbb műszer által mért érték nincs egyszerű egyenes arányosságban a koncentrációval, ami megbízhatatlan értéket eredményez. Célszerű ilyenkor a $5 - 10 \times$ higítást végezni, és újramérni az abszorbanciát, majd megszorozni a higítással a koncentrácóra kapott értéket.
A $t = \infty$ reakcióidőhöz tartozó termékkoncentrációkat, amelyek megfelelnek az \emph{Aspirin} kezdeti koncentrációjának, igen nagy reakcióidőnél vett mintából lehet meghatározni.
A méréseket két hőmérsékleten kell végrehajtani, ezeket a gyakorlatvezető határozza meg a gyakorlat kezdetén.
A reakció Arrhenius paramétereinek meghatározása érdekében ajánlott hőmérséklet 313 és 353 K.

2 db \emph{Aspirin} tablettát külön-külön dörzsmozsárban elporítunk, és külön-külön főzőpohárban kevés desztillált vízben oldunk, majd 25 cm$^3$-es mérőlombikokba szűrjük és jelig töltjük (I. és II. törzsoldat).
%A törzsoldatokon kívül még az alábbi oldatokat kell elkészíteni: 50-50 cm3 térfogatú 0.25 M HCl, 0.25 M NaOH, és 0.1 M HCl-ben oldott 0.1 M FeCl3.

Reakció alacsonyabb hőmérsékleten, pl. 313-323 K-en (a gyakorlatvezető határozza meg):

a) Az Aspirin kezdeti koncentrációjának (z) meghatározása:
Az I. sz. törzsoldatból 2 cm3 mintát csiszolatos dugós Erlenmeyer lombikba pipettázunk, hozzáadunk 3 cm3 0.25 M NaOH oldatot és a lombikot belehelyezzük a pontosan ismert hõmérsékletû termosztátba. 
A 60. percben a reakciót befagyasztjuk (a lombikot jeges vízbe és kevés desztillált vízzel mossuk, atot és desztillált vízzel 100 cm3-re hígítjuk.


b) A t idõpillanatig elbomlott Aspirin (x) koncentrációjának meghatározása:
Az I. törzsoldat maradékát a mérõlombikból csiszolatos dugós Erlenmeyer lombikba töltjük át (nem mossuk!), termosztátba helyezzük (t=0 perc), és a lombik kivétele nélkül a bomlás 15, 20, 25, 30 és 35. percében 2 cm3-es mintákat veszünk, amelyet az elõkészített 25 cm3-s mérõlombikokba töltünk.
A mintákhoz hozzáadunk 0.5 cm3 0.25 M HCl-at, 0.5 cm3 0.1 M FeCl3-at és 25 cm3 össztérfogatra hígítjuk õket desztillált vízzel.


Reakció magasabb hõmérsékleten, pl. 333-343 K-en (a gyakorlatvezetõ határozza meg)


a) Az Aspirin kezdeti koncentrációjának meghatározása:
A II. sz. törzsoldatból 2 cm3-t csiszolatos dugós Erlenmeyer lombikba pipettázunk, hozzáadunk 0.6 cm3 0.25 M NaOH-t és 15 cm3 desztillált vizet, majd a lombikot belehelyezzük az ismert hõmérsékletû termosztátba.
A 60. percben a reakciót befagyasztjuk, mintához lombik tartalmát 100 cm3-es mérõlombikba töltjük és kevés desztillált vízzel átmossuk, s a mintához hozzáadunk 1 cm3 0.25 M HCl-t, 2 cm3 0.1 M FeCl3-at, majd desztillált vízzel 25 cm3-re hígítjuk.

b) A t idõpontig elbomlott Aspirin koncentrációjának meghatározása:
A II. törzsoldat maradékát a mérõlombikból csiszolatos dugós Erlenmeyer lombikba töltjük át nem mossuk!), termosztátba helyezzük (t = 0 perc) és a lombik kivétele nélkül a bomlás 10, 15, 20, 25 és 30. percében 2 cm mintákat veszünk.
A 10, 15 és 20. percben vett mintákhoz hozzáadunk 0.5 cm3 0.25 M HCl-t és 0.5 cm3 0.l mólos FeCl3-at és 25 cm3-re hígítjuk desztillált vízzel.
Figyelem!
A kísérleteket célszerû úgy végezni, hogy a megadott térfogatú mérõlombikba elõre bepipettázzuk a megfelelõ mennyiségû sósavat és a FeCl3 oldatot, a lombikot 2/3-ad részéig feltöltjük desztillált vízzel, s jeges vízbe állítjuk. 
Az összes minta fogadására szükséges mérõlombikot készítsük így elõ és tegyük õket jeges vízbe.


Fényabszorpció mérése


Mind a kezdeti, mind a t idõpillanatban lévõ koncentráció meghatározása spektrofotometriásan történik.
A spektrofotométer kezelési leírása a készülék mellett megtalálható.


\subsection{Beadandó eredmények}

1. táblázat: A mérési és számított adatok táblázatosan

2. táblázat: A sebességi együttható hőmérsékletfüggése

3. A sebességi állandó hõmérsékletfüggésébõl határozzuk meg a sebességi állandó értékét 20 °C-on (293 K) grafikusan, ábrázolva a lg k - az l/T függvényében.

4. Az Arrhenius egyenlet integrált alakjába történõ behlyettesítéssel számtsuk ki az E aktiválási energiát és a preexponciális tényezõt:
E [kJ mol-1]
lgA[s-1]
A[s-1]

*standard deviációs számítása
s=….

\end{document}
