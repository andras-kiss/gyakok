\documentclass{article}
\usepackage{amsmath}
\usepackage{amssymb}
\usepackage{gensymb}
\usepackage{upgreek}
\usepackage{float}
\usepackage{graphicx}

\begin{document}

\title{Important information on the Physical Chemistry laboratory practice for pharmacy students}
%\author{Andr\'{a}s Kiss, assistant lecturer}
%Department of General and Physical Chemistry, Faculty of Sciences, University of P\'{e}cs, 7624 P\'{e}cs, Ifj\'{u}s\'{a}g \'{u}tja 6, Hungary
\date{2016.09.06}
\maketitle
%\address[akiss, gnagy]{J\'{a}nos Szent\'{a}gothai Research Centre, University of P\'{e}cs, 7624 P\'{e}cs, Ifj\'{u}s\'{a}g \'{u}tja 20, Hungary}
%\ead{akiss@gamma.ttk.pte.hu}

The semester consists of 12 laboratory practices, and two tests. For each practice, thorough preparation is recommended. Before each practice, you must pass a small entrance test, answering a few questions about the topic. If you cannot answer these questions sufficiently, you cannot start the practice, and have to do a do-over on the 14th week. During the practice, you must record the measured data in your laboratory notebook with a non-erasable pen. Additionally, after finishing the practice, you must evaluate the data, calculate the required results, and submit it by an agreed deadline on every week. You must use your own data, and no-one else's. Each practice's notebook will be graded on a 0 - 5 point scale. At the end of the semester, a total of 60 points can be gathered from notebooks. 

You will write the two tests: on the 7th, and on the 13th week. You can gather 12-12 points from these tests by answering the questions for all 6 practices, including the one you will do on the day of the test.

You need to reach at least 6 points (50\%) in both tests to finish the course. You also need to have at least 1 point for each practice to have it accepted. If you do the practice, but fail to evaluate it, you will get zero points for that practice.

You can miss one practice, but then you have to make it up for it with a do-over on the 14th week. You can re-write the tests as well.

To summarize: 

\begin{enumerate}
\item 60 points from practices + notebooks (12 $\times$ 5 points). Need to have at least 1 point for each.
\item 12 points from the first test. 50\% necessary.
\item 12 points from the second test. 50\% necessary. 
\end{enumerate}

If you meet all these requirements, you will get a final grade based on the sum of your points:

\begin{center}
\begin{tabular}{|c|c|}
\hline
0 -- 41 & 1 (insufficient) \\
\hline
52 -- 53 & 2 (satisfactory) \\
\hline
54 -- 63 & 3 (average) \\
\hline
64 -- 73 & 4 (good) \\
\hline
74 -- 84 & 5 (excellent) \\
\hline
\end{tabular}
\end{center}

The tests will be written on the Friday of the 7th week (2016.10.21) and the 13th week (2016.12.02) on Friday


\end{document}
